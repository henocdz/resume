%%%%%%%%%%%%%%%%%%%%%%%%%%%%%%%%%%%%%%%%%
% Developer CV
% LaTeX Template
% Version 1.0 (28/1/19)
%
% This template originates from:
% http://www.LaTeXTemplates.com
%
% Authors:
% Jan Vorisek (jan@vorisek.me)
% Based on a template by Jan Küster (info@jankuester.com)
% Modified for LaTeX Templates by Vel (vel@LaTeXTemplates.com)
%
% License:
% The MIT License (see included LICENSE file)
%
%%%%%%%%%%%%%%%%%%%%%%%%%%%%%%%%%%%%%%%%%

%----------------------------------------------------------------------------------------
%   PACKAGES AND OTHER DOCUMENT CONFIGURATIONS
%----------------------------------------------------------------------------------------

\documentclass[9pt]{developercv} % Default font size, values from 8-12pt are recommended

\usepackage[utf8]{inputenc}
%----------------------------------------------------------------------------------------

\begin{document}

%----------------------------------------------------------------------------------------
%   TITLE AND CONTACT INFORMATION
%----------------------------------------------------------------------------------------

\begin{minipage}[t]{0.45\textwidth} % 45% of the page width for name
    \vspace{-\baselineskip} % Required for vertically aligning minipages
    \colorbox{black}{{\HUGE\textcolor{white}{\textbf{\MakeUppercase{Henoc}}}}}
    
    \colorbox{black}{{\HUGE\textcolor{white}{\textbf{\MakeUppercase{Diaz}}}}}
    \vspace{6pt}
\end{minipage}
\begin{minipage}[t]{0.275\textwidth} % 27.5% of the page width for the first row of icons
    \vspace{-\baselineskip} % Required for vertically aligning minipages

    % The first parameter is the FontAwesome icon name, the second is the box size and the third is the text
    % Other icons can be found by referring to fontawesome.pdf (supplied with the template) and using the word after \fa in the command for the icon you want
    \icon{MapMarker}{12}{Mexico}\\
    \icon{Linkedin}{12}{\href{https://www.linkedin.com/in/henocdz/}{/henocdz}}\\
    \icon{At}{12}{\href{mailto:henocdz@gmail.com}{henocdz@gmail.com}}\\
\end{minipage}
\begin{minipage}[t]{0.275\textwidth} % 27.5% of the page width for the second row of icons
    \vspace{-\baselineskip} % Required for vertically aligning minipages

    % The first parameter is the FontAwesome icon name, the second is the box size and the third is the text
    % Other icons can be found by referring to fontawesome.pdf (supplied with the template) and using the word after \fa in the command for the icon you want
    \icon{Globe}{12}{\href{https://henoc.dev}{henoc.dev}}\\
    \icon{Github}{12}{\href{https://github.com/henocdz}{/henocdz}}\\
\end{minipage}

\vspace{0.5cm}

%----------------------------------------------------------------------------------------
%   About
%----------------------------------------------------------------------------------------

\begin{minipage}[t]{1\textwidth} % 40% of the page width for the introduction text
    \cvsect{Who's this}
    {  \\
        I'm a software engineer and manager with a passion for creating impactful products 
        that positively transform people's lives. My motivation lies in products that drive fundamental 
        change and innovation. I'm mostly interested in working with platform or product engineering 
        teams using the cloud and innovative technologies. \\

        I have a diverse background and experience working with Go, Python, JavaScript along with 
        cloud-native technologies such as AWS, Docker, Terraform and GCP. I have a strong interest 
        in software design and architecture, product, and mentoring. \\ 
    
        With over 8 years of experience, having been an individual contributor, a former engineering 
        leader, and a mentor, I've helped organizations and individuals to design, build, and scale 
        their engineering teams, products, processes, and culture by designing, building and crafting 
        scalable and secure processes and software solutions. 
    }
\end{minipage}


%----------------------------------------------------------------------------------------
%   EXPERIENCE
%----------------------------------------------------------------------------------------

\cvsect{Experience}

\begin{entrylist}
    \entry
        {
            Apr 2023 -- Present
            \\\footnotesize{Mexico (Remote)}
        }
        {Senior Software Engineer}
        {Apli ({\href{https://apli.jobs/}{\underline{apli.jobs}}})}
        {
            \lorem \lorem\\

            \texttt{Software Design \& Architecture}\slashsep\texttt{Reliability \& Performance}\slashsep\texttt{Python}\slashsep\texttt{AWS}\slashsep\texttt{Elastic Search}\slashsep\texttt{SQL (Postgres)}\slashsep\texttt{MongoDB}\slashsep\texttt{Microservices}
        }

    \entry
        {Oct 2018 -- Apr 2023\\\footnotesize{Mexico (Remote)}}
        {CTO}
        {Apli ({\href{https://apli.jobs/}{\underline{apli.jobs}}})}
        {
            \lorem \lorem\\

            \texttt{People Management}\slashsep\texttt{Product Management}\slashsep\texttt{Project Management}\slashsep\texttt{Product Design}\slashsep\texttt{Engineering Management}\slashsep\texttt{Leadership}\slashsep\texttt{Software Design \& Architecture}\slashsep\texttt{Python}\slashsep\texttt{Go}\slashsep\texttt{SQL (Postgres)}\slashsep\texttt{MongoDB}\slashsep\texttt{Microservices}
        }

    \entry
        {
            Jan -- Oct 2018
            \\\footnotesize{Mexico City}
        }
        {Tech Lead}
        {Apli ({\href{https://apli.jobs/}{\underline{apli.jobs}}})}
        {

            Led a five-developers team. Responsible for breaking down software projects, previously scoped by the CTO, and their prioritization and assignment to the engineers; conducted daily SCRUM-like status updates. 

            Actively participated in weekly catch-ups with C-level management (CEO, CSO, CTO, COO) regarding the engineering team's progress.

            Fully responsible for AWS Infrastructure while the CTO was on vacation or unavailable.

            \texttt{Management}\slashsep\texttt{Python}\slashsep\texttt{Javascript (React \& Node)}\slashsep\texttt{AWS}\slashsep\texttt{Elastic Search}\slashsep\texttt{SQL (Postgres)}\slashsep\texttt{MongoDB}\slashsep\texttt{Microservices}
        }

    \entry
        {
            Nov 2017 -- Jan 2018
            \\\footnotesize{Mexico City}
        }
        {Senior Software Engineer}
        {Apli ({\href{https://apli.jobs/}{\underline{apli.jobs}}})}
        {
            Responsible for designing software solutions for internal operations teams along with intense bug hunty nights and some fuck ups

            \texttt{Management}\slashsep\texttt{Python}\slashsep\texttt{Javascript (React \& Node)}\slashsep\texttt{AWS}\slashsep\texttt{Elastic Search}\slashsep\texttt{SQL (Postgres)}\slashsep\texttt{MongoDB}\slashsep\texttt{Microservices}
        }

    \entry
        {
            May -- Nov 2017
            \\\footnotesize{Mexico City}
        }
        {Front-end Developer}
        {Inventive ({\href{https://www.linkedin.com/company/inventivehack/}{\underline{inventivehack.com}}})}
        {
            Led frontend projects for customers like Boletia, Punkpost, and Custodian. Responsible for quality and good practices, along with tasks definition. Collaborated with customers' engineering teams on defining and breaking down features.\\
            
            \texttt{ReactJS}\slashsep\texttt{AWS (S3)}

        }

    \entry
        {
            Jun 2016 -- \\Apr 2017
            \\\footnotesize{Mexico City}
        }
        {Full-stack Developer}
        {Sable ({\href{https://sable.mx/}{\underline{sable.mx}}})}
        {
            Provided guidance on software development to exceptional UI and UX designers as they implemented software solutions using Ruby on Rails, JavaScript, and CSS. Learned a lot regarding UI/UX topics and taught good software practices. 

            Also, participated as a freelancer software developer for some of Sable's clients. The main client was Gestionix (acquired by Konfío, 2020), for whom I was responsible for the engineering team in charge of designing, and implementing Gestionix's Accountants platform; participated in customer meetings for prioritization and specs gathering as well.\\
            \texttt{Python}
            \slashsep\texttt{Ruby}
            \slashsep\texttt{React}
            \slashsep\texttt{Digital Ocean}
        }

    \entry
        {
            Sept 2015 --\\Apr 2017
            \\\footnotesize{Mexico City}
        }
        {Software Developer}
        {Asistia ({\href{https://www.linkedin.com/company/asistia/}{\underline{asistia.mx}}})}
        {
            First developer of the team, responsible for designing and implementing Asistia's web platform bare bones. Responsible for designing and implementing a React Native application for Android devices. 

            Collaborated in SCRUM-like biweekly sprints, helping to delimit, prioritize, and estimate tasks complexity.
            
            Briefly led a two-developers team. Rejected an offer to become the CTO but stayed at the company to learn from the hired CTO.\\
            \texttt{Python (Django + Celery)}
            \slashsep\texttt{React Native}
            \slashsep\texttt{Digital Ocean}}

    \entry
        {
            Jan -- Dec 2015
            \\\footnotesize{Mexico City}
        }
        {Course instructor \& Software developer}
        {Dev.F ({\href{https://devf.la/}{\underline{devf.la}}})}
        {
            A key member of the committee in charge of the development and design of DevF's education program for 2015 – including all levels, from basic to advanced software development. 

            Mentored 6 student batches (+50 students total) – teaching topics ranging from basic to advanced software development.
            
            Responsible for designing, coding, and deploying DevF's very first payment system which continued to be used for around two years after I left.\\
            \texttt{Python}
            \slashsep\texttt{Public speaking}
        }

    \entry
        {
            Jan -- Aug 2015
            \\\footnotesize{Mexico City}
        }
        {Software Developer}
        {Angel Ventures Mexico ({\href{http://www.angelventures.vc/}{\underline{angelventures.vc}}})}
        {
            Worked on Meridia's (Angel Ventures funded) internal system with previously delimited and defined features. 

            Implemented WordPress-based landing pages for Meridia's subsidiaries.\\
            \texttt{Ruby}
            \slashsep\texttt{Wordpress}
            \slashsep\texttt{Node.js}
        }

\end{entrylist}

%----------------------------------------------------------------------------------------
%   EDUCATION
%----------------------------------------------------------------------------------------

\cvsect{Education}

\begin{entrylist}
    \entry
        {
            2011 -- 2016
            \\\footnotesize{(Drop-out)}
        }
        {BS - Computer Systems Engineering}
        {IPN - Escuela Superior de Cómputo}
        {Completed 80\% of credits.}
\end{entrylist}

\cvsect{Other}

\begin{entrylist}
    \entry
        {2015 -- 2016}
        {Co-Founder, Co-organizer \& Speaker}
        {
            Chilango Django \\
            \texttt{Public speaking}\slashsep\texttt{Community management}
        }
        {
            
            Co-founded and co-organized the largest meetup for Python and Django in 
            Mexico City with {\href{https://meetup.com/Chilango-Django/}{\underline{+1.5k}}} members. \\

            \small{Talks: Celery, Python Mocking, Introduction to Python \& Django}
            
        }
    \entry
        {2012\\\footnotesize{Online}}
        {Teacher}
        {Codigo Facilito {\href{(https://codigofacilito.com/}{(codigofacilito.com)}}}
        {
            Recorded 
            {\href{https://www.youtube.com/watch?v=CjmzDHMHxwU&list=PLE549A038CF82905F}{\underline{Python 2.7}}} 
            and {\href{https://www.youtube.com/watch?v=jKbjblt4NXA&list=PLpOqH6AE0tNi47LF-_6gddgq10lp_TLDB}{\underline{Javascript + jQuery}}} 
            video-courses. Both with thousands of students across Latin America.\\
            \texttt{Video edition}
        }
\end{entrylist}

\cvsect{Languages}

\begin{minipage}[t]{1\textwidth}
    \vspace{-\baselineskip} % Required for vertically aligning minipages
    \textbf{Spanish} - native
    \slashsep\textbf{English} - proficient
    \slashsep\textbf{Italian} - learning
\end{minipage}


\end{document}