%%%%%%%%%%%%%%%%%%%%%%%%%%%%%%%%%%%%%%%%%
% Developer CV
% LaTeX Template
% Version 1.0 (28/1/19)
%
% This template originates from:
% http://www.LaTeXTemplates.com
%
% Authors:
% Jan Vorisek (jan@vorisek.me)
% Based on a template by Jan Küster (info@jankuester.com)
% Modified for LaTeX Templates by Vel (vel@LaTeXTemplates.com)
%
% License:
% The MIT License (see included LICENSE file)
%
%%%%%%%%%%%%%%%%%%%%%%%%%%%%%%%%%%%%%%%%%

%----------------------------------------------------------------------------------------
%   PACKAGES AND OTHER DOCUMENT CONFIGURATIONS
%----------------------------------------------------------------------------------------

\documentclass[9pt]{developercv} % Default font size, values from 8-12pt are recommended

\usepackage[utf8]{inputenc}
%----------------------------------------------------------------------------------------

\begin{document}

%----------------------------------------------------------------------------------------
%   TITLE AND CONTACT INFORMATION
%----------------------------------------------------------------------------------------

\begin{minipage}[t]{0.45\textwidth} % 45% of the page width for name
    \vspace{-\baselineskip} % Required for vertically aligning minipages
    \colorbox{black}{{\HUGE\textcolor{white}{\textbf{\MakeUppercase{Henoc}}}}}

    \colorbox{black}{{\HUGE\textcolor{white}{\textbf{\MakeUppercase{Diaz}}}}}
    \vspace{6pt}
\end{minipage}
\begin{minipage}[t]{0.275\textwidth} % 27.5% of the page width for the first row of icons
    \vspace{-\baselineskip} % Required for vertically aligning minipages

    % The first parameter is the FontAwesome icon name, the second is the box size and the third is the text
    % Other icons can be found by referring to fontawesome.pdf (supplied with the template) and using the word after \fa in the command for the icon you want
    \icon{MapMarker}{12}{Mexico}\\
    \icon{Linkedin}{12}{\href{https://www.linkedin.com/in/henocdz/}{/henocdz}}\\
    \icon{At}{12}{\href{mailto:henocdz@gmail.com}{henocdz@gmail.com}}\\
\end{minipage}
\begin{minipage}[t]{0.275\textwidth} % 27.5% of the page width for the second row of icons
    \vspace{-\baselineskip} % Required for vertically aligning minipages

    % The first parameter is the FontAwesome icon name, the second is the box size and the third is the text
    % Other icons can be found by referring to fontawesome.pdf (supplied with the template) and using the word after \fa in the command for the icon you want
    \icon{Globe}{12}{\href{https://henoc.dev}{henoc.dev}}\\
    \icon{Github}{12}{\href{https://github.com/henocdz}{/henocdz}}\\
\end{minipage}

\vspace{0.5cm}

%----------------------------------------------------------------------------------------
%   About
%----------------------------------------------------------------------------------------

\cvsect{Who's this}

\begin{minipage}[t]{1\textwidth} % 40% of the page width for the introduction text
    {  
        I'm a software engineer and manager with a passion for creating impactful products 
        that positively transform people's lives. My motivation lies in products that drive fundamental 
        change and innovation. I'm mostly interested in working with platform and product engineering 
        teams using the cloud and innovative technologies. \\

        I have a diverse background and experience having worked with Go, Python, JavaScript, Postgres, 
        Elastic Search, MongoDB, and cloud-native technologies such as AWS, Docker, Terraform, and 
        GCP. I have a strong interest in software design and architecture, product, and mentoring. \\ 
    
        With over 8 years of experience, having been an individual contributor, a former engineering 
        leader, and a mentor, I've helped organizations and individuals to design, build, and scale 
        their engineering teams, products, processes, and culture by designing, building and crafting 
        scalable and secure processes and software solutions. 
    }
\end{minipage}


%----------------------------------------------------------------------------------------
%   EXPERIENCE
%----------------------------------------------------------------------------------------

\cvsect{Experience}

\begin{entrylist}
    \entry
        {
            Apr 2023 -- Present
            \\\footnotesize{Mexico (Remote)}
        }
        {Senior Software Engineer – Apli ({\href{https://apli.jobs/}{\underline{apli.jobs}}}) }
        {
            \texttt{Software Design \& Architecture}
            \slashsep\texttt{Reliability \& Performance}
            \slashsep\texttt{Python}
            \slashsep\texttt{Terraform}
            \slashsep\texttt{AWS}
            \slashsep\texttt{Postgres}
            \slashsep\texttt{Microservices}
        }
        {\\
            After deciding to leave my CTO role, I decided to stay at Apli as a Senior Software 
            Engineer to continue growing as an engineer and to help the company with the transition.\\

            I reduced the API response time by 10x by leveraging Postgres optimizations such as 
            indexes and cluster configuration.\\

            I also led and contributed to the design and implementation of an auto-posting project
            leveraging event-driven architecture and AWS services such as Batch, SQS, and Lambda. 
            This project also leveraged RPA (Robotic Process Automation) to automate job postings on
            Job Board websites.
        }

    \entry
        {
            Oct 2018 -- Apr 2023
            \\\footnotesize{Mexico (Remote)}
        }
        {CTO – Apli ({\href{https://apli.jobs/}{\underline{apli.jobs}}}) }
        {
            \texttt{Python}
            \slashsep\texttt{Go}
            \slashsep\texttt{Javascript (React)}
            \slashsep\texttt{Terraform}
            \slashsep\texttt{AWS}
            \slashsep\texttt{Postgres}
            \slashsep\texttt{DynamoDB}
            \slashsep\texttt{Elastic Search}
            \slashsep\texttt{MongoDB}
            \slashsep\texttt{Software Design \& Architecture}
            \slashsep\texttt{Cloud Infrastructure}
            \slashsep\texttt{Microservices}
            \slashsep\texttt{People Development \& Management}
            \slashsep\texttt{Product \& Project Management}
            \slashsep\texttt{Engineering Leadership}
        }
        {\\
            I was a very hands-on CTO, leading the engineering team in regards to people management,
            organization processes, cross-team collaboration and I was also responsible for the 
            design and implementation of some of the company's software solutions.\\

            In regards to people management, I was responsible for the hiring, onboarding, and
            performance reviews of the engineering team in order to help them grow - including 
            the design of each of these processes. I also designed the engineering career ladder 
            which helped to keep a low rotation rate - this initiative was later adopted by the 
            whole company.\\

            In regards to organization processes, I proposed and led product and project 
            management initiatives across the company such as weekly catch-ups with other leaders, 
            recurrent all-hands meetings, bi-weekly bugs review meetings, and the implementation and 
            evangelization of productivity tools such as ClickUp, Slack integrations, Retool, 
            and others. \\

            Additionally, enforced engineering processes such as code reviews, pull requests, 
            on-call rotations, DORA metrics, and others.
        }

    \entry
        {}
        {}
        {}
        {\\
            Regarding software design and architecture, I was responsible for the design and
            implementation of some of the company's critical software solutions such as:

            \begin{itemize}
                \item Side effects: a system to trigger configurable actions based on changes in some database tables. Leveraging techniques such as polling and asynchronous execution to achieve high performance and reliability.
                \item Recruitment Pipeline: the main system capability responsible for managing the whole recruitment process. This lego-like software solution was designed to be highly configurable and extensible and has been used by the company to manage over 3M candidates with highly customizable processes demanded by customers such as BBVA, Grupo Bimbo, and others.
                \item Openings Locator: a system leveraging Postgres' natural language capabilities to allow chat-users to search for locations where there are openings available in a conversational style.
                \item Job Boards Automation Messaging: a system contact candidates that applied via Job Board websites to continue their process via Apli's chatbot. Notification are highly configurable allowing N number of retries, and different messaging platforms such as Email, WhatsApp and SMS. This system leveraged event-driven architecture, RPA (Robotic Process Automation), and AWS services such as Batch, DynamoDB, SQS, and Lambda.
                \item Conversational platform: led the engineering product team responsible for the product and technical design of Apli's core chatbot platform in record time (< 6 months). Also contributed with this system's Conversation Flow State capability taking inspiration from data structures such as stacks and queues to allow the chatbot to handle complex conversations scenarios. This capability has been used by over 10M conversations and was highly optimized at Postgres level to achieve < 500ms response times since the chatbot user sends a message until the chatbot responds.
                \item Infrastructure: designed and implemented the first version of Apli's infrastructure on AWS for both platforms (Conversations and Recruitment). The first version was designed to be highly available and scalable, and was later migrated to a more robust and secure infrastructure. 
                \item CI \& CD: designed and implemented continuous integration and continuous delivery pipelines using Team City and GitHug Actions. This allowed the engineering team to deploy to production multiple times a day with confidence.
            \end{itemize}

            In regards to security, I led and contributed to the company's efforts to achieve the
            ISO 27001 certification. This included the design and implementation of security
            processes, policies, and software solutions. I was responsible for defining the 
            company's security posture and corresponding policies in regard to device management, 
            access control, software development, and infrastructure. This included the deployment
            of MDM, AV, and cloud security solutions such as Kandji, Crowdstrike, AWS Security Hub, 
            and others.\\

            Before leaving this role, I redesigned the engineering organization structure to better 
            align with the company's long-term goals and to set the foundation to scale the 
            engineering team.
        }

    \entry
        {
            Jan -- Oct 2018
            \\\footnotesize{Mexico City}
        }
        {Tech Lead – Apli ({\href{https://apli.jobs/}{\underline{apli.jobs}}})}
        {
            \texttt{Python}
            \slashsep\texttt{Javascript (React \& Node)}
            \slashsep\texttt{AWS}
            \slashsep\texttt{Postgres}
            \slashsep\texttt{Project Management}
        }
        {\\

            I led a five-developers team and I was responsible for breaking down software projects, 
            previously delimited by the CTO, as well as their prioritization and assignment to my 
            engineering peers. I also led daily SCRUM-like meetings and weekly catch-ups with C-level
            management.\\
            
            I was also responsible for AWS Infrastructure monitoring and maintenance.
        }

    \entry
        {
            Nov 2017 -- Jan 2018
            \\\footnotesize{Mexico City}
        }
        {Senior Software Engineer – Apli ({\href{https://apli.jobs/}{\underline{apli.jobs}}})}
        {
            \texttt{Management}
            \slashsep\texttt{Python}
            \slashsep\texttt{Javascript (React \& Node)}
            \slashsep\texttt{AWS}
            \slashsep\texttt{Elastic Search}
            \slashsep\texttt{SQL (Postgres)}
            \slashsep\texttt{MongoDB}
            \slashsep\texttt{Microservices}
        }
        {\\
            Responsible for designing software solutions for internal operations teams along with 
            leading bug hunting efforts.
        }

    \entry
        {
            May -- Nov 2017
            \\\footnotesize{Mexico City}
        }
        {Front-end Developer – Inventive ({\href{https://www.linkedin.com/company/inventivehack/}{\underline{inventivehack.com}}})}
        { 
            \texttt{ReactJS}
            \slashsep\texttt{AWS (S3)}
        }
        {\\
            I led frontend projects for customers like Boletia, Punkpost, and Custodian. I was 
            responsible for the quality, good practices by the team, along with tasks definition.\\
            
            I was also responsible for the design and implementation of the frontend architecture
            and collaborated closely with the company customer's engineering teams to set goals and 
            roadmaps.\\
        }

    \entry
        {
            Jun 2016 -- Apr 2017
            \\\footnotesize{Mexico City}
        }
        {Full-stack Developer – Sable ({\href{https://sable.mx/}{\underline{sable.mx}}})}
        {
            \texttt{Python}
            \slashsep\texttt{Ruby}
            \slashsep\texttt{ReactJS}
            \slashsep\texttt{Digital Ocean}
        }
        {\\
            I led the engineering team in charge of designing and implementing Gestionix's 
            (Sable's customer, acquired by Konfío in 2020) Accountants platform, and collaborated with 
            the UI and UX designers on the frontend (ReactJS).\\

            I also collaborated closely with the UI and UX teams to implement software solutions 
            using Ruby on Rails, JavaScript, and CSS.\\
        }

    \entry
        {
            Sept 2015 -- Apr 2017
            \\\footnotesize{Mexico City}
        }
        {Software Developer – Asistia ({\href{https://www.linkedin.com/company/asistia/}{\underline{asistia.mx}}})}
        {
            \texttt{Python (Django + Celery)}
            \slashsep\texttt{React Native}
            \slashsep\texttt{Digital Ocean}
        }
        {\\
            Set the foundation for Asistia's software platform by designing and implementing the a 
            RESTful API, using Django and Celery, a React Native App (Android only), and internal 
            operations tools.\\

            Collaborated in SCRUM-like biweekly sprints, helping to delimit, prioritize, and estimate 
            tasks complexity.\\
            
            Briefly led a two-developers team. Rejected an offer to become the CTO but stayed at the company to learn from the hired CTO.
        }

    \entry
        {
            Jan -- Dec 2015
            \\\footnotesize{Mexico City}
        }
        {Course instructor \& Software developer – Dev.F ({\href{https://devf.la/}{\underline{devf.la}}})}
        {
            \texttt{Python}
            \slashsep\texttt{Public speaking}
        }
        {\\
            Co-designed DevF's education program for 2015, from basic to intermediate software development. \\

            Mentored +50 students, teaching topics ranging from basic to intermediate software 
            development such as Python, Django, Git, and Linux. \\
            
            As a software developer I was designed and implemented DevF's very first payment system 
            which continued to be used until 2018.
        }

    \entry
        {
            Jan -- Aug 2015
            \\\footnotesize{Mexico City}
        }
        {Software Developer – Angel Ventures Mexico ({\href{http://www.angelventures.vc/}{\underline{angelventures.vc}}})}
        {   
            \texttt{Ruby}
            \slashsep\texttt{Wordpress}
            \slashsep\texttt{Node.js}
        }
        {\\
            Worked on a portfolio's company (Meridia) internal system using Ruby on Rails with 
            previously delimited and defined features. \\

            Implemented WordPress-based landing pages for Meridia's subsidiaries.
        }

\end{entrylist}

%----------------------------------------------------------------------------------------
%   EDUCATION
%----------------------------------------------------------------------------------------

\cvsect{Education}

\begin{entrylist}
    \entry
        {
            2011 -- 2016
            \\\footnotesize{(Drop-out)}
        }
        {BS - Computer Systems Engineering}
        {IPN - Escuela Superior de Cómputo}
        {Completed 80\% of credits.}
\end{entrylist}

\cvsect{Other}

\begin{entrylist}
    \entry
        {2015 -- 2016}
        {Co-Founder, Co-organizer \& Speaker}
        {
            Chilango Django \\
            \texttt{Public speaking}\slashsep\texttt{Community management}
        }
        {
            
            Co-founded and co-organized the largest meetup for Python and Django in 
            Mexico City with {\href{https://meetup.com/Chilango-Django/}{\underline{+1.5k}}} members. \\

            \small{Talks: Celery, Python Mocking, Introduction to Python \& Django}
            
        }
    \entry
        {2012\\\footnotesize{Online}}
        {Teacher}
        {Codigo Facilito {\href{(https://codigofacilito.com/}{(codigofacilito.com)}}}
        {
            Recorded 
            {\href{https://www.youtube.com/watch?v=CjmzDHMHxwU&list=PLE549A038CF82905F}{\underline{Python 2.7}}} 
            and {\href{https://www.youtube.com/watch?v=jKbjblt4NXA&list=PLpOqH6AE0tNi47LF-_6gddgq10lp_TLDB}{\underline{Javascript + jQuery}}} 
            video-courses. Both with thousands of students across Latin America.\\
            \texttt{Video edition}
        }
\end{entrylist}

\cvsect{Languages}

\begin{minipage}[t]{1\textwidth}
    \vspace{-\baselineskip} % Required for vertically aligning minipages
    \textbf{Spanish} - native
    \slashsep\textbf{English} - proficient
    \slashsep\textbf{Italian} - learning
\end{minipage}



\cvsect{Favorite Books}

\begin{minipage}[t]{1\textwidth}
    \vspace{-\baselineskip} % Required for vertically aligning minipages
    \textbf{Build}, \textit{by Tony Fadell} \\
    \textbf{The Phoenix Project}, \textit{by Gene Kim et al.} \\
    \textbf{Think Like a Rocket Scientists}, \textit{by Ozan Varol} \\
\end{minipage}


\end{document}